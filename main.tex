%% start of file `template.tex'.
%% Copyright 2006-2013 Xavier Danaux (xdanaux@gmail.com).
%
% This work may be distributed and/or modified under the
% conditions of the LaTeX Project Public License version 1.3c,
% available at http://www.latex-project.org/lppl/.


\documentclass[12pt,a4paper,roman,colorlinks,linkcolor=true]{moderncv}        % possible options include font size ('10pt', '11pt' and '12pt'), paper size ('a4paper', 'letterpaper', 'a5paper', 'legalpaper', 'executivepaper' and 'landscape') and font family ('sans' and 'roman')

% modern themes
\moderncvstyle{banking}                            % style options are 'casual' (default), 'classic', 'oldstyle' and 'banking'
\moderncvcolor{black}                                % color options 'blue' (default), 'orange', 'green', 'red', 'purple', 'grey' and 'black'
%\renewcommand{\familydefault}{\sfdefault}         % to set the default font; use '\sfdefault' for the default sans serif font, '\rmdefault' for the default roman one, or any tex font name
\nopagenumbers{}                                  % uncomment to suppress automatic page numbering for CVs longer than one page

% character encoding
\usepackage[utf8]{inputenc}
\usepackage{fontawesome}
\usepackage{tabularx}
\usepackage{ragged2e}
% if you are not using xelatex ou lualatex, replace by the encoding you are using
%\usepackage{CJKutf8}                              % if you need to use CJK to typeset your resume in Chinese, Japanese or Korean

% adjust the page margins
\usepackage[scale=0.8]{geometry}
\usepackage{multicol}
%\setlength{\hintscolumnwidth}{3cm}                % if you want to change the width of the column with the dates
%\setlength{\makecvtitlenamewidth}{10cm}           % for the 'classic' style, if you want to force the width allocated to your name and avoid line breaks. be careful though, the length is normally calculated to avoid any overlap with your personal info; use this at your own typographical risks...

\usepackage{import}

% personal data
\name{Samuel E.}{Bassi}
% \title{Curriculum Vitae}                               % optional, remove / comment the line if not wanted
\address{Esteban Echeverría 36, Wilde, Buenos Aires}{}{}% optional, remove / comment the line if not wanted; the "postcode city" and and "country" arguments can be omitted or provided empty
% \phone[mobile]{909-839-3097}                   % optional, remove / comment the line if not wanted
% \phone[fixed]{01234 123456}                    % optional, remove / comment the line if not wanted
%\phone[fax]{+3~(456)~789~012}                      % optional, remove / comment the line if not wanted
% \email{xpan1@swarthmore.edu}                               % optional, remove / comment the line if not wanted
% \homepage{shawnpan.me}                         % optional, remove / comment the line if not wanted
% \extrainfo{}                 % optional, remove / comment the line if not wanted
%\photo[64pt][0.4pt]{picture}                       % optional, remove / comment the line if not wanted; '64pt' is the height the picture must be resized to, 0.4pt is the thickness of the frame around it (put it to 0pt for no frame) and 'picture' is the name of the picture file
%\quote{Some quote}                                 % optional, remove / comment the line if not wanted

% to show numerical labels in the bibliography (default is to show no labels); only useful if you make citations in your resume
%\makeatletter
%\renewcommand*{\bibliographyitemlabel}{\@biblabel{\arabic{enumiv}}}
%\makeatother
%\renewcommand*{\bibliographyitemlabel}{[\arabic{enumiv}]}% CONSIDER REPLACING THE ABOVE BY THIS

% bibliography with mutiple entries
%\usepackage{multibib}
%\newcites{book,misc}{{Books},{Others}}
  
\newcommand*{\customcventry}[7][.25em]{
  \begin{tabular}{@{}l} 
    {\bfseries #4}
  \end{tabular}
  \hfill% move it to the right
  \begin{tabular}{l@{}}
     {\bfseries #5}
  \end{tabular} \\
  \begin{tabular}{@{}l} 
    {\itshape #3}
  \end{tabular}
  \hfill% move it to the right
  \begin{tabular}{l@{}}
     {\itshape #2}
  \end{tabular}
  \ifx&#7&%
  \else{\\%
    \begin{minipage}{\maincolumnwidth}%
      \small#7%
    \end{minipage}}\fi%
  \par\addvspace{#1}}

\newcommand*{\customcvproject}[4][.25em]{
%   \vfill\noindent
  \begin{tabular}{@{}l} 
    {\bfseries #2}
  \end{tabular}
  \hfill% move it to the right
  \begin{tabular}{l@{}}
     {\itshape #3}
  \end{tabular}
  \ifx&#4&%
  \else{\\%
    \begin{minipage}{\maincolumnwidth}%
      \small#4%
    \end{minipage}}\fi%
  \par\addvspace{#1}}

\setlength{\tabcolsep}{12pt}

%----------------------------------------------------------------------------------
%            content
%----------------------------------------------------------------------------------
\begin{document}
\hypersetup{urlcolor=blue}
%\begin{CJK*}{UTF8}{gbsn}                          % to typeset your resume in Chinese using CJK
%-----       resume       ---------------------------------------------------------

\makecvtitle
\vspace*{-23mm}

\begin{center}
\begin{tabular}{ c c c c }
 %\faGlobe\enspace sebport0.github.io
 \faEnvelopeO\enspace 
 \href{mailto:sebassi@live.com}{sebassi@live.com} & \faGithub\enspace \href{https://github.com/sebport0}{@sebport0} &
 \faLinkedin\enspace
 \href{https://www.linkedin.com/in/samuel-bassi-85a8bba3/}{Samuel Bassi} & 
 \faMobile\enspace 15 5948 5409\\  
\end{tabular}
\end{center}

\section{FORMACIÓN ACADÉMICA}
{\customcventry{2016 - presente}{Licenciatura en Ciencias Matemáticas}{Facultad de Ciencias Exactas y Naturales, UBA}{CABA}{}{}}

{\customcventry{2007 - 2012}{Técnico Electrónico}{E.T. N 1 Otto Krause}{CABA}{}{}}

%\section{KEY SKILLS}

\section{PROYECTOS}

{\customcvproject{Introducción a la Neurociencia Cognitiva y Computacional}{2018}
  {\begin{itemize}
    \item Replicamos el experimento realizado en "Unconscious  Addition:  When  we  unconsciously initiate and follow arithmetic rules."(Ric \& Muller, 2012). Mediante la interacción con el programa, el usuario debe realizar una suma de forma inconsciente.  Repo en github: \href{https://github.com/Maunsi/Neurociencia}{github.com/Maunsi/Neurociencia}.
    \item Python, PsychoPy, Pandas, Numpy, Matplotlib.
  \end{itemize}
  }
}

{\customcvproject{freeCodeCamp}{2019}
{\begin{itemize}
  \item Serie de proyectos orientados al desarrollo web. Entre ellos: drum machine, reloj pomodoro, app del clima. Repo en github: \href{https://github.com/sebport0/fcc-projects}{github.com/sebport0/fcc-projects}.
  \item Javascript, React, Redux, D3, HTML, CSS, Node, Express, MongoDB. 
\end{itemize}
}

% % {\customcvproject{Amazing Project 3}{Jan 2015 – May 2015}
% % {\begin{itemize}
% %   \item Assist in recruiting potential and admitted students to the UIUC computer science program
% %   \item Attend information sessions, admitted student Q\&A’s, and student lunches to generate interest in the Illinois Computer Science department
% % \end{itemize}
% % }
% }

\section{CURSOS}
  
  \begin{minipage}{\maincolumnwidth}
    \small{
      \begin{itemize}
          \item freeCodeCamp JavaScript Algorithms and Data Structures, 2019.\\ Certificado: \href{https://www.freecodecamp.org/certification/sebport0/javascript-algorithms-and-data-structures}{freecodecamp.org/sebport0}
      \end{itemize}}
  \end{minipage}
%\section{AWARDS AND ACHIEVEMENTS}
%\begin{minipage}{\maincolumnwidth}%
%	\small{
%    	\begin{itemize}
%          \item Winner, Tata Innovista, November 2016
%          \item President of XYZ Club, Public Relations Manager of Swarthmore QWE Club
%          \item Programming Languages: Python, C, C++, PHP, Java, HTML/CSS, Javascript, jQuery, NodeJS
%          \item Fluent in Gibberish, conversational in Nonsense
%		\end{itemize}}%
%\end{minipage}%


\section{EXPERIENCIA LABORAL}

%{\customcventry{June 2016 - May 2017}{Technology Analyst}{Tata Consultancy Servies Limited}{Kolkata, India}{}
%{\begin{itemize}
%  \item Designed and built the first version of infrastructure code base which built the entire environment of SDLC at runtime. This reduced the development teams’ monthly expenses on AWS servers by 100 thousand dollars.
%  \item Designed and a centralized build strategy for all the software components of our client. The monolithic design of our build module brought new business to TCS
%\end{itemize}
%}

%}

{\customcventry{2015 - presente}{Suplencia Docente}{Colegio N7 DE3, Juan M. de Pueyrredón}{CABA}{}
{\begin{itemize}
    \item Clases de Físico - Química para tercer año del turno noche.
    \item Clases de Matemática Financiera para quinto año del turno tarde.
    \item Clases de Física para cuarto año del turno noche.
    \item Clases de Matemática para exámenes previos.
\end{itemize}
}
}

\section{CONOCIMIENTOS}
  \textbf{Informáticos:} Javascript, HTML/CSS, React, C++, Python, Git, \LaTeX. \\
  \textbf{Idiomas:} Inglés(\textit{intermedio}).




%\section{AWARDS AND ACHIEVEMENTS}
%\begin{minipage}{\maincolumnwidth}%
%	\small{
%    	\begin{itemize}
%          \item Winner, Tata Innovista, November 2016
%          \item President of XYZ Club, Public Relations Manager of Swarthmore QWE Club
%          \item Programming Languages: Python, C, C++, PHP, Java, HTML/CSS, Javascript, jQuery, NodeJS
%          \item Fluent in Gibberish, conversational in Nonsense
%		\end{itemize}}%
%\end{minipage}%
      
}
% Publications from a BibTeX file without multibib
%  for numerical labels: \renewcommand{\bibliographyitemlabel}{\@biblabel{\arabic{enumiv}}}% CONSIDER MERGING WITH PREAMBLE PART
%  to redefine the heading string ("Publications"): \renewcommand{\refname}{Articles}
\nocite{*}
\bibliographystyle{plain}
\bibliography{publications}                        % 'publications' is the name of a BibTeX file

% Publications from a BibTeX file using the multibib package
%\section{Publications}
%\nocitebook{book1,book2}
%\bibliographystylebook{plain}
%\bibliographybook{publications}                   % 'publications' is the name of a BibTeX file
%\nocitemisc{misc1,misc2,misc3}
%\bibliographystylemisc{plain}
%\bibliographymisc{publications}                   % 'publications' is the name of a BibTeX file

%-----       letter       ---------------------------------------------------------

\end{document}


%% end of file `template.tex'.
